\section{Következtetések}
A célunk egy olyan módszer megalkotása volt, amely az adott beállítású elektrofiziológiai kísérletek eredményeinek megbízhatóságát, pontosságát képes jellemezni. Ennek megvalósítása érdekében a Bayesiánus-valószínűség alapköveit használtuk fel. Így elértük, hogy a bemenetként szolgáltatott kísérleti adatok, idegsejtmodell és becsülendő paraméterei, valamint zajmodell mellett képesek vagyunk a keresett paraméterek tartományához egy poszterior eloszlást társítani. Ez az eloszlás pedig jellemzi az adott kísérleti protokollból kinyerhető információtartalmat (a keresett paraméterekre nézve). Ebben a dolgozatban azt vizsgáltuk, hogy tényleg jól működik-e ez a megközelítés. Ezt zajmodellek segítségével kreált szintetikus kísérleti adatokat használva tettük.

Láttuk, hogy több paramétert együtt mérve, valamint a realisztikusabb színes zajjal romlik a becslés pontossága. Azt is megfigyeltük, hogy a kiterjedt modell dendritre jellemző passzív paraméterét, az axiális ellenállást ($Ra$) kevésbé pontosan becsültük tisztán a sejttesten végzett mérések alapján. Ez is egy bizonyíték arra, hogy a módszerünk tényleg alkalmas a kísérleti protokoll információtartalmának jellemzésére. Valós kísérletekből nyert adatok és (komplex) morfológia esetén is elkezdtük a vizsgálódást fehér zajmodellel közelítve a kísérleti zajt. Ennek a következő lépése, hogy autokorrelációt végezve a színes zajon meghatározzuk annak kovarianciamátrixát és ezzel végezzük el az inferenciát (\ref{meth:dependet}). Az eredmények azt mutatják, hogy érdemes ezzel a megközelítéssel foglalkozni és továbbfejleszteni azt.

Technikai oldalról jelenleg egy "brute-force" megvalósítás zajlik\footnote{A program forrása itt fellelhető: \href{https://github.com/terbed/parameter-inference}{https://github.com/terbed/parameter-inference}.}, azaz a paraméterek tartományain (lineárisan mintavételezve) numerikus integrálás folyik. Ez kevés paraméter esetén jól működik, viszont ha nagyszámú változóra szeretném alkalmazni exponenciálisan lelassul (valós kísérleti adatsoron végzett háromparaméteres becslés is órás nagyságrendbe esett a Python multiprocessing lehetőségeit is kihasználva és egy 2.6 GHz Intel Core i7-es processzort alkalmazva). Ezért érdemes lenne ezt  továbbfejleszteni a paraméterteret "intelligensebben" bejáró módszerekkel. Láttuk (\ref{app:sampling}) hogy priorból mintavételezve elegendő volt kevesebb kiértékelési pont jó eredményhez, ehhez hasonlóan például a Markov chain Monte Carlo (MCMC) módszerrel sokkal jobban közelíthetőek az eredmények nagy dimenzió esetén. Határestet vizsgáló közelítő módszer a folytonos eloszlásokra a Laplace módszer \cite{azevedo1994laplace}, amivel pontosabban jellemezni lehetne a kapott poszterior eloszlást. Ezt is érdemes lenne beépíteni a jövőben.

Végső cél realisztikus zajmodellek segítségével szintetikus adatokat előállítva megbecsülni, hogy adott kísérleti protokollból mennyi információ nyerhető ki, így segítve a kísérletezők munkáját. Itt pedig már elengedhetetlen az előbb tárgyalt közelítő módszerek alkalmazása.



\section*{Köszönetnyilvánítás} 
Először is szeretném megköszönni a témavezetőmnek, Dr. Káli Szabolcsnak a rengeteg szakértelmét, útmutatását és türelmes hozzáállását ami nélkül ez a dolgozat nem jöhetett volna létre. Továbbá szeretném megköszönni a barátaim és családtagjaim megértését, ösztönzését mialatt a dolgozattal foglalkoztam.