\section{Következtetések}
A célunk egy olyan módszer megalkotása volt, amely az adott beállítású elektrofiziológiai kísérletek eredményeinek megbízhatóságát, pontosságát képes jellemezni. Ennek megvalósítása érdekében a Bayesiánus-valószínűség alapköveit használtuk fel. Így elértük, hogy a bemenetként szolgáltatott kísérleti adatok, idegsejtmodell és becsülendő paraméterei, valamint zajmodell mellett képesek vagyunk a keresett paraméterek tartományához egy poszterior eloszlást társítani. Ez az eloszlás pedig jellemzi az adott kísérleti protokollból kinyerhető információtartalmat (a keresett paraméterekre nézve). Ebben a dolgozatban azt vizsgáltuk, hogy tényleg jól működik-e ez a megközelítés. Ezt zajmodellek segítségével kreált szintetikus kísérleti adatokat használva tettük.

Láttuk, hogy több paramétert együtt mérve, valamint a realisztikusabb színes zajjal romlik a becslés pontossága. Azt is megfigyeltük, hogy a kiterjedt modell dendritre jellemző passzív paraméterét, az axiális ellenállást ($Ra$) kevésbé pontosan becsültük tisztán a sejttesten végzett mérések alapján. Ez is egy bizonyíték arra, hogy a módszerünk tényleg alkalmas a kísérleti protokoll információtartalmának jellemzésére.


\section*{Köszönetnyilvánítás} 
Először is szeretném megköszönni a témavezetőmnek, Dr. Káli Szabolcsnak a rengeteg szakértelmét, útmutatását és türelmes hozzáállását ami nélkül ez a dolgozat nem jöhetett volna létre. Továbbá szeretném megköszönni a barátaim és családtagjaim megértését, ösztönzését mialatt a dolgozattal foglalkoztam.