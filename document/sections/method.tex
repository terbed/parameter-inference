\section{Módszerek és megvalósítás}

\subsection{Eloszlásbecslés Bayesiánus formalizmussal}
Célunk az idegsejt modellezés eszközeivel adott mérési eredmények alapján különböző paraméterek poszterior eloszlását meghatározni a kiválasztott sejttípuson belül, mivel ez sok lényeges információt hordozhat (paraméterek várható értéke, közöttük levő korreláció, becslés információtartalma...). Erre a problémakörre jól alkalmazható a Baysiánus inferencia módszere, amit a következőkben tárgyalunk.

\paragraph{valószínűségi változók}
A modell paramétereire úgy tekintünk, mint valószínűségi változókra $(\xi)$, melyekhez eloszlásfüggvényt szeretnénk rendelni.

\paragraph{poszterior eloszlás}
Paraméterek poszterior eloszlása szeretnénk meghatározni adott kísérleti eredmények mellett.
Elsődleges szempont tehát, hogy legyenek kísérleti eredményeink (például a mi esetünkben passzív idegsejtek válasza áramimpulzus hatására). Valamint jellemezni kell valamilyen módon, hogy az adatok mennyire támasztják alá a modellünk adott paraméterek melletti helyességét. Ezeken felül hasznos, ha vannak előzetes ismereteink a becsülendő paraméterekről. Ezen összetevőkből egy poszterior eloszlás készíthető a következő módon:

\begin{equation}\label{eq:bayes}
P_i(\xi|D) = \dfrac{P_i(D|\xi)P(\xi)}{\int P_i(D|\xi)P(\xi)d\xi}
\end{equation}
ahol $D$ a kísérleti adat, $i$ az áramimpulzusra adott választ jelöli és $\xi$ pedig a modellparamétereinket.
\begin{itemize}
	\item $P_i(\xi|D)$: Ez a poszterior eloszlás, a keresett paraméterek valószínűségi eloszlása a mérési adatok figyelembevétele után.
	\item $P_i(D|\xi)$: Ez a likelihood eloszlás, azt jellemzi mennyire valószínű ezeknek az adatoknak a mérése, ha az adott paraméterbeállítással vett modellünket vesszük igaznak. A következő pontban ezt részletesen tárgyaljuk.
	\item $P(\xi)$: Ez a prior eloszlás, előzetes ismereteink a paraméterekről. Úgy is felfoghatjuk, hogy ezt az eloszlást frissítjük az új adatok függvényében, így keletkezik a poszterior eloszlás. Új méréseket végezve ez a folyamat tovább iterálható.
	\item $\int P_i(D|\xi)P(\xi)d\xi$: Ez csupán a normálási faktor. Annak a következménye, hogy a valószínűségi eloszlásoknak normáltnak kell lenniük, így teljes tartományra vett integráljuknak egy. A normálási faktor ezzel a tulajdonsággal ruházza fel a poszterior értékünket, aminek következtében teljes értékű valószínűségi eloszlás lesz.
\end{itemize}

\paragraph{marginalizálás}
Előfordulhat, hogy néhány paraméterre nem vagyunk kíváncsiak (például csak azért vettük be a modellünkbe, hogy lássuk mennyire torzítja el a többi paraméter eloszlását). Tegyük fel, hogy $\xi$ a cél paramétereink, de a modellt kiterjesztettük további $\theta$ változóval, aminek viszont az eloszlása nem érdekel. Továbbra is $P_i(\xi|D)$ meghatározása a feladat.  A likelihood viszont ekkor ilyen formában írható fel:
\[
P_i(D|\xi, \theta)
\]
A "felesleges" változókat egyszerűen kiintegrálva (a priorjaival együtt) visszakapjuk a~\ref{eq:bayes}-es egyenletben szereplő likelihood formát:

\begin{eqnarray}\label{eq:marginalize}
P_i(D|\xi) = \int P_i(\xi, \theta)P(\theta)d\theta
\end{eqnarray}
Viszont ez nem feltétlen fogja ugyan azt az eredményt szolgáltatni, ugyanis a paraméterek között komplex összefüggések lehetnek.

\paragraph{numerikus implementálás}
Numerikusan nem lehetséges egzaktul folytonos függvények kezelése. Mégis diszkrét eloszlások bevezetése helyett, \textit{kvázi-folytonosnak} tekintjük őket és alkalmazzuk rá a numerikus eljárásokat, mintha folytonosak lennének.

\subsection{Maximum likelihood módszer}

\subsection{Zajmodellek}