\documentclass[hidelinks, 12pt]{article}
\usepackage{amsmath}
\usepackage{listings}
\usepackage{gensymb}
\usepackage[magyar]{babel}
\usepackage[utf8]{inputenc}
\usepackage[a4paper, margin=3cm ]{geometry}
\usepackage{graphicx}
\usepackage{subfig}
\usepackage{siunitx}
\usepackage[T1]{fontenc}
\usepackage{lmodern}
\usepackage{booktabs}
\usepackage{textcomp}
\usepackage{color}
\usepackage{hyperref}
\hypersetup{colorlinks=false}

\usepackage{placeins}
\renewcommand{\baselinestretch}{1.13}
\numberwithin{equation}{subsection}
\numberwithin{figure}{subsection}
\numberwithin{table}{subsection}

\newcommand{\Lagr}{\mathcal{L}}
\newcommand{\N}{\mathcal{N}}
\newcommand{\M}{\mathcal{M}}
\let\vaccent=\v % \vaccent mostantól \v
\renewcommand{\v}[1]{\mathbf{#1}} %félkövér vektor
\newcommand{\gv}[1]{\mbox{\boldmath$ #1 $}} % félkövér vektor görög betűvel
\newcommand{\uv}[1]{\mathbf{\hat{#1}}} % félkövér egységvektor kalappal
\newcommand{\abs}[1]{\left| #1 \right|} % abszolut érték
\newcommand{\avg}[1]{\left< #1 \right>} % átlag
\let\underdot=\d % a\underdot mostantól \d
\renewcommand{\d}[2]{\frac{d #1}{d #2}} % derivált
\newcommand{\dd}[2]{\frac{d^2 #1}{d #2^2}} % második derivált
\newcommand{\pd}[2]{\frac{\partial #1}{\partial #2}} %parciális derivált
\newcommand{\pdd}[2]{\frac{\partial^2 #1}{\partial #2^2}} % második parciális derivált
\newcommand{\pddv}[3]{\frac{\partial^2 #1}{\partial #2 \partial #3}} % második vegyes parciális derivált
\newcommand{\pdc}[3]{\left. \frac{\partial #1}{\partial #2} \right|_{#3}} % Termodinamikai parc. derivált
\newcommand{\mpdc}[3]{\left. \frac{\partial^2 #1}{\partial #2^2} \right|_{#3}} %Második termodinamikai parc. derivált
\newcommand{\mvpdc}[3]{\left.\left. \frac{\partial ^2 #1}{\partial #2 \partial #3}\right|_{#2}\right|_{#3}}
\newcommand{\ket}[1]{\left| #1 \right>} % bra
\newcommand{\bra}[1]{\left< #1 \right|} % ket
\newcommand{\braket}[2]{\left< #1 \vphantom{#2} \right|
	\left. #2 \vphantom{#1} \right>} % braket
\newcommand{\matrixel}[3]{\left< #1 \vphantom{#2#3} \right| #2 \left| #3 \vphantom{#1#2} \right>} % Szendvics
\newcommand{\grad}[1]{\gv{\nabla} #1} % gradiens
\let\divsymb=\div % mostantól a \divsymb-> \div
\renewcommand{\div}[1]{\gv{\nabla} \cdot #1} % divergencia
\newcommand{\rot}[1]{\gv{\nabla}\times #1} % rotáció
\newcommand{\cint}[4]{\int _#1^#2 #3 d\v{#4}} %felületi/görbementi integrál
\newcommand{\vint}[4]{\int _#1^#2 #3 d^3\v{#4}}%térfogati integrál
\newcommand{\crossp}[2]{\v{#1}\times\v{#2}}%keresztszorzat sima betűkkel
\newcommand{\gcrossp}[2]{\gv{#1}\times\v{#2}}%keresztszorzat, az első komponens görög
\newcommand{\levi}[1]{\varepsilon _{#1}}%Levi-Civita
\newcommand{\kron}[1]{\delta _{#1}}%Kronecker delta
\newcommand{\mmat}[1]{\underline{\underline{{#1}}}}%Mátrix
\newcommand{\gmmat}[1]{\underline{\underline{{#1}}}}%Görög mátrix
\newcommand{\szum}[2]{\sum \limits _{#1} ^{#2}} %Szumma határokkal
\newcommand{\pois}[2]{\left\lbrace {#1},{#2}\right\rbrace} %Poisson-zárójel