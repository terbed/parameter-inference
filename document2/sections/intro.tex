\section{Bevezetés}

Elektrofiziológiai kísérletek során, az idegsejtet jellemző bizonyos paramétereket nehézkesen, csak közvetett módon lehet mérni. Ezért az a bevett módszer, hogy egy idegsejt bizonyos áramstimulusra adott válaszát kísérleti úton rögzítjük és azt replikálni próbáljuk a sejtről készített kompartmentumos modell ugyanazon (programozott) áramimpulzusra adott szimulációs válaszával. Kezdetben az intuíció erejével kézzel állítgatták a paramétereket. Ma is vannak erre kifejlesztett kreatív technikák \cite{eichner2011hands}. A módszer azon alapszik, hogy adott paraméterbeállítással kiértékeljük a modell eredményét és összehasonlítjuk annak kísérleti adatokkal vett hasonlóságát. Ezt addig ismételgetjük, míg elfogadható hasonlóságot nem mutat a kettő. Viszont ahogy a számítógépek egyre hatékonyabbá és gyorsabbá váltak, lehetőség nyílott a fentebbi folyamat automatizálására, ahol már a modellünk adott paraméterek melletti teljesítményét, \textit{"jóságát"} egy algoritmus értékeli ki a hibafüggvény segítségével. Utóbbi hasonlítja össze a modell -- adott paraméterek melletti -- eredményét a kísérleti adatokkal és egy számértékkel jellemzi az eltérésüket. Az illesztési procedúra eredménye egy megoldás a keresett paraméterekre, kiegészítve az így kapott modell hibájával. Ezeknek a megvalósítására rengeteg módszer van és széles körben elterjedtek, sok cikk foglalkozik velük \cite{druckmann2007novel}\cite{van2007neurofitter}\cite{van2008automated}. Külön megemlíteném az -- ugyancsak Káli Szabolcs vezetésével megalkotott --  \textit{Optimizer}\cite{friedrich2014flexible} nevű idegsejt paraméter optimalizációs szoftvert, amely több algoritmust használva és kombinálva kiemelkedő teljesítmény nyújt a területén, már csak a felhasználóbarát grafikus kezelőfelülete miatt is.

Ennek a megközelítésnek viszont sok hátulütője lehet, ugyanis a zaj miatt -- melyet az előbb tárgyalt paraméteroptimalizációs eljárások nem vesznek figyelembe -- a sejtválasz ugyanarra az áramstimulusra kísérletről-kísérletre változhat. Ezen kívül nem ad elegendő betekintést a módszer arra sem, hogy mennyire pontos a paraméterek becslése, illetve milyen kapcsolat lehet közöttük.

Olyan is előfordulhat, hogy a valóditól teljesen eltérő paraméterértékeket tudunk becsülni a kísérleti adatokból. Például az elektromosan csatolt neurális hálózatoknál is ez a helyzet\cite{amsalem2016neuron}\cite{szoboszlay2016functional}, mivel a csatoltság kihatással van a sejtet leíró passzív paraméterekre és megváltoztatja azok tényleges értékét. Ennek következtében az a megoldás, miszerint megkeressük a paramétereket melyekkel a sejtmodellünk tökéletesen visszaadja a kísérleti eredményeket, félrevezető lehet. De ez már nem mondható az optimalizációs módszer hibájának, ügyelni kell, hogy ne legyenek csatoltak a hálózatok, ha tényleges paramétereket akarunk mérni.

A zajból és a becslés pontosságának bizonytalanságából fakadó probléma orvoslására egy valószínűségi modellt hoztunk létre a Bayesiánus formalizmust használva, ami eredményül egy poszterior eloszlást társít a keresett paraméterekhez (mint valószínűségi változókhoz), az adott kísérleti adatok függvényében. Ezzel ellenben az optimalizációs algoritmusok csupán egyetlen értéket adnak egy hibaértékkel társítva megoldásként. Ez a módszer lehetőséget ad arra, hogy a kísérletből származó bizonytalanságot is figyelembe tudjuk venni, ami egy kiterjedtebb, részletesebb jellemzése az egész problémakörnek. Több különböző területen már sikeresen alkalmazták ezt a megközelítést, például a szinaptikus paraméterek megbecslésére\cite{costa2013probabilistic}. A cikkből kiderül, hogy sikeresen alkalmazták a módszert arra -- ami számunkra igen releváns --, hogy a kísérleti protokollt javítsák a nagyobb információtartalmú paraméterbecslés érdekében. Emellett kiderült, hogy a különböző szélességű eloszlások jellemzik a különböző szinaptikus dinamikákat, így ennek segítségével azok csoportosíthatók. 


Tehát elektrofiziológiai mérések idegsejtmodellekkel való kiegészítése nem újkeletű, de a statisztikai vizsgálódás -- szemben a paraméter optimalizációs algoritmusokkal -- kevésbé elterjedt, az idegtudomány területén eddig még nem is nagyon alkalmazták. Tehát a mi általunk tárgyalt megközelítés célja nem csupán az, hogy találjunk egy olyan paraméterkombinációt a modellhez, ami legjobban illeszkedik a kísérleti adatsorra, hanem hogy valamilyen módon jellemezzük az adott kísérleti összeállítás információtartalmát. Ez azért fontos mivel a paraméterek különböző megválasztásával azok becslése is különböző hatékonysággal történik, mert köztük bonyolult összefüggések, korrelációk lehetnek, így az eredmény pontosságának szempontjából nem mindegy milyen beállításokkal végezzük a kísérletet. Még mielőtt az adott kísérleti protokollal elvégeznék a mérést, mi szeretnénk megmondani -- szintetikus, \textit{"kvázi-kísérleti"} adatokat előállítva és arra inferenciát végezve --, hogy az mennyi információt képes szolgáltatni a keresett biofizikai paraméterekről. Ez egy nagyon hasznos, sok helyen felhasználható, általános eredmény lenne a kísérletek megtervezése szempontjából. Ráadásul az elektrofiziológiai kísérletek költségesek és véges idő áll rendelkezésre a sejten végzett mérésekhez, ezért fontos lenne hatékony méréseket kivitelezni, például megmondani, hogy akkor lesz a legeredményesebb a kísérlet, ha \textit{A} típusú stimulussal mérnek \textit{x} darabot és \textit{B} típusú stimulussal mérnek \textit{y} darabot stb...


A munkánk során egyszerű statisztikai modellt állítunk fel, amit különböző komplexitású modellek passzív paramétereinek eloszlásának becslésére alkalmazunk, különböző zajmodellek mellett. Végeredményként a paraméterekre mint valószínűségi változókra egy eloszlást fogunk kapni, ami képes jellemezni a mérés bizonytalanságát. Végül arra jutunk, hogy a módszer jól működik és kiterjeszthető a fő problémára: adott kísérleti protokoll információtartalmának előrejelzésére, valamint el is kezdjük néhány protokoll vizsgálatát.


A következőkben röviden áttekintjük, hogy hogyan modellezzük az idegsejteket és lesz szó a \textit{NEURON} idegsejt modellező programról is, amit a munkánk során alkalmaztunk. Ezután rátérünk a konkrétumokra: fentebb leírt feladat megvalósítására, eredményekre.



