\section{Következtetések}
A célunk egy olyan módszer megalkotása volt, amely az adott beállítású elektrofiziológiai kísérletek eredményeinek megbízhatóságát, pontosságát képes jellemezni. Ennek megvalósítása érdekében a Bayesiánus-valószínűség alapköveit használtuk fel. Így elértük, hogy a bemenetként szolgáltatott kísérleti adatok, idegsejtmodell és becsülendő paraméterei, valamint zajmodell mellett képesek vagyunk a keresett paraméterek tartományához egy poszterior eloszlást társítani. Ez az eloszlás pedig jellemzi az adott kísérleti protokollból kinyerhető információtartalmat (a keresett paraméterekre nézve). Ebben a dolgozatban azt vizsgáltuk, hogy tényleg jól működik-e ez a megközelítés és alkalmazható-e kísérleti protokollok előzetes jellemzésére.

Láttuk, hogy több paramétert együtt mérve, valamint a realisztikusabb színes zajjal romlik a becslés pontossága. Továbbá megmutattuk, hogy a módszerünk alkalmas a paraméterek közti bonyolult kapcsolatok feltérképezésére. Azt is megfigyeltük, hogy a kiterjedt modell dendritre jellemző passzív paraméterét, az axiális ellenállást ($Ra$) kevésbé pontosan becsültük tisztán a sejttesten végzett mérések alapján. Ez is egy bizonyíték arra, hogy a módszerünk tényleg alkalmas a kísérleti protokoll információtartalmának jellemzésére. Ezek után épp erre tértünk rá, vagyis a különböző stimulusból álló protokollok információtartalmát határoztuk meg az egyes paraméterekre nézve. Az eredmények teljesen egybevágtak az intuíciónkkal, ismét alátámasztva a módszert. Végül valós kísérletekből nyert adatok és bonyolult morfológia esetére is végeztünk inferenciát, fehér zajmodellel közelítve a kísérleti zajt. Ezen kívül vizsgáltuk a kísérlet során előforduló zajt is, de a rendelkezésre álló adatok nem voltak elegendőek, hogy zajmodellt hozzunk létre.

Technikai oldalról jelenleg egy \textit{"brute force"} megvalósítás zajlik\footnote{A program forrása itt fellelhető: \href{https://github.com/terbed/parameter-inference}{https://github.com/terbed/parameter-inference}.}, azaz a paraméterek tartományain -- lineárisan, vagy a priorból mintavételezve -- numerikus integrálás folyik. Ez kevés paraméter esetén jól működik, viszont ha nagyszámú változóra szeretném alkalmazni, exponenciálisan lelassul. Három paraméter esetén nagyobb felbontás mellett 14 órába telt az utolsó ábrákat (\ref{fig:result}) legenerálni, egy korszerű négymagos processzorral ellátott gépen futtatva a programot. Ezért érdemes lenne továbbfejleszteni ezt, a paraméterteret \textit{"intelligensebben"} bejáró módszerekkel. Ilyen például a \textit{Markov chain Monte Carlo (MCMC)} módszer, amely elterjedt közelítő módszer nagy dimenziószámú szimulációk esetén. Egy másik közelítő módszer a \textit{Laplace módszer}\cite{azevedo1994laplace}, amely folytonos normál eloszlással közelíti az eredményeket második deriváltakat használva. Ehhez elegendő néhány -- megfelelően megválasztott -- mintavételi pont a paramétertérből. Ezt is érdemes lenne beépíteni a jövőben.

Végső cél realisztikus zajmodellek segítségével szintetikus adatokat előállítva megbecsülni, hogy adott kísérleti protokollból mennyi információ nyerhető ki, így segítve a kísérletezők munkáját. Ennek érdekében különböző alakú stimulusokat kezdtünk tesztelni. Többek között különböző frekvenciájú szinuszos árambemeneteket. Azt is elkezdtük vizsgálni, hogy érdemesebb-e több különböző szinusz komponenst mérni és azokat kombinálni, vagy jobb-e, ha egy fajtából mérünk sokat. Az a cél, hogy a kísérletezőknek ajánlani tudjunk mérési terveket, azaz hogy a rendelkezésükre álló idő alatt milyen típusú protokollokból mennyit mérjenek annak érdekében, hogy a maximális információtartalmat kihozhassák a mérésből.

Végül fontos megemlíteni, hogy a módszer teljesen általános (protokoll tervezés). Mi elektrofiziológiai kísérletekre és passzív idegsejtekre alkalmaztuk, de akár általánosítható aktív idegsejtekre, illetve teljesen más jellegű mérésekre is. Az inferencia módszere még általánosabb: hogy jellemezni tudjuk a paraméterbecslés pontosságát ugyanis csak mérési adatok és az illesztendő modell kell, a mérés során előforduló zaj ismeretével karöltve. Ebben az esetben a dolgozatban ismertetett módszer segítségével sokkal mélyrehatóbban lehet jellemezni a becslés pontosságát, mint a paraméteroptimalizációs algoritmusok által szolgáltatott illesztési hibával.