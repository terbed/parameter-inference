\renewcommand{\abstractname}{Kivonat}
\begin{abstract}
\

Az idegsejtek viselkedését számos paraméter határozza meg, melyek közül kísérletileg nem mérhető mind közvetlenül. Az ilyen típusú paraméterekre a kísérletből csupán közvetett módon következtethetünk. Ezért modelleket alkalmazunk bizonyos paraméterek kísérleti eredményekből való meghatározására.\


A munkánk célja az, hogy valószínűségszámítás módszereivel, illetve a modellek szimulációjával meghatározzuk, hogy adott kísérleti összeállítások segítségével mennyire pontosan vagyunk képesek a paraméterek becslésére. A Bayesiánus valószínűségszámítás alapvető módszereit és a modellek szimulációs eredményeit felhasználva kiszámíthatjuk, hogy adott kísérleti eredményeket milyen paraméterek mekkora valószínűséggel adják.
\

Ezeket a módszereket először egyszerű esetekben teszteltük. Megállapítottuk, hogy egy egykompartmentumos passzív idegsejt membránparamétereit fehér zaj jelenlétében becsülni tudjuk az idegsejtmodell áramlépcső bemenetre adott válasza alapján. Kezdetben az egykompartmentumos modell membrán kapacitását választottuk valószínűségi változónak. Majd ugyanezt azzal kiegészítve, hogy a passzív konduktanciát is változónak vettük. A következő lépésben térbelileg kiterjedt modellek axiális ellenállását és passzív membrán konduktanciáját választottuk a becsülendő paramétereknek. Majd ezeket az eseteket általánosítottuk exponenciálisan korreláló színes zajra, mert ez egy realisztikusabb modellje a kísérleti zajoknak. Végül vizsgáltuk, hogy a különböző kísérleti protokollok használata hogyan befolyásolja az egyes paraméterek becslését, ezzel alapozva a végső célt: a legnagyobb információtartalmú protokoll ajánlása a kísérletezőknek.\

Végeredményként arra jutottunk, hogy a becslés pontossága függ a becsülni kívánt paraméterek számától és azok összeállításától, valamint az egyes kísérleti összeállításoktól is. Tehát bizonyos paramétereket együtt mérve az egyes paraméterek értékéről kevesebb információt szerzünk, mint amennyit esetleges más összeállításokból kinyerhetnénk, valamint egyfajta idegsejt stimulálás esetén az egyik paramétert pontosabb becsülni, míg másik esetén a másikat. A gyakorlatban fontos speciális esetként megfigyeltük, hogy a kiterjedt modell dendritre jellemző paraméterei (pl. az axiális ellenállás) kevésbé pontosan mérhetők tisztán a sejttesten végzett mérések alapján.\

Összességében megállapíthatjuk, hogy az általunk kidolgozott paraméterbecslési módszerek alkalmasak arra, hogy a kísérleti adatok alapján megbecsüljük ne csak önmagában a legvalószínűbb paraméterértékeket, hanem maga az inferencia várható pontosságát és akár a paraméterek korrelációját is.\

Legfőbb feladatunk, hogy sok különböző paraméterösszeállítást használva szintetikus adatokat állítunk elő, melyekre aztán alkalmazzuk a paraméterbecslést. Ennek segítségével képesek leszünk előre megmondani, hogy az adott kísérletet elvégezve mennyire pontos eredményt kapnánk, mekkora lenne a mérés információtartalma.
\end{abstract}